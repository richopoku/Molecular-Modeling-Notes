\documentclass[a4paper,11pt]{report}
\usepackage[utf8]{inputenc}
\usepackage{graphicx}
\usepackage{float}
\usepackage{amsmath}
\usepackage{amssymb}
\usepackage{natbib}
\graphicspath{{images/}}
\usepackage[width=150mm,top=15mm,bottom=25mm]{geometry}
\usepackage{times}
\usepackage{textcomp}
\usepackage{caption}
\usepackage{booktabs}
\renewcommand{\baselinestretch}{2}
\renewcommand{\figurename}{\textbf{Figure}}
\renewcommand{\contentsname}{Table of Contents}
\renewcommand{\bibname}{References}
\renewcommand{\baselinestretch}{2}
\renewcommand\bibname{References}


\thispagestyle{empty}



\begin{document}
%\maketitle
%\chapter*{}

\pagenumbering{roman}

%\chapter*{}

%\chapter*{}

%\chapter*{}

%\tableofcontents
%\listoffigures
%\listoftables
%\chapter{Introduction}
\pagenumbering{arabic}



\chapter{Theoretical Background}
\section{Density Functional Theory (DFT}
\subsection{The origin of DFT}
The main idea of the DFT \citep{jensen2017introduction} method is based on the avoidance the calculation of the many-electron system wave function
to provide solution to the Schrodinger equation. It describes the many-electron system through the density of wave function ($\Psi$), where the
electron density is considered to be dependent on 3N space coordinates for a N-electron system instead of the wave function
$\Psi(r_{1},r_{2},...,r_{N})$ which is a much more complex quantity and is given as
\begin{equation}\label{3nspace}
\rho(r) = N\sum_{\sigma_{1}}\int\bigg|\psi(r_{1}\sigma_{1},...,r_{N}\sigma_{N})\bigg|^{2}dr_{2}...dr_{N}d\sigma_{2}...d\sigma_{N}
\end{equation}
\section{The Hohenberg-Kohn Theorem}
The Hohenberg-Kohn theorem \citep{hohenberg1964inhomogeneous} gives theoretical justification of the above assumption whereby the ground-state
wave function of a 
system of electrons is treated as a functional of its ground-state electron density $\big(\psi_{o}[\rho]\big)$. More vividly, it is considered that, in
the many-electron system there exist one-to-one mapping between the external potential, $V_{ext}$ and the ground-state electron density with
an interparticle interaction $w(r_{i},r_{j})$.
Therefore, the total energy as a functional of the electron
density constructed from the ground-state electron density, $\rho_{o}$ and the ground-state energy, $E_{o}$ is expressed as
\begin{equation}\label{eground}
 E[\rho(r)] = \big \langle\psi_{o}[\rho]\big|\hat{T}+\hat{V}_{en}+\hat{V}_{ee}\big|\psi_{o}[\rho]\big \rangle = \int V(r)\rho(r)dr + F_{HK}[\rho]
 \end{equation}
where $V(r)$ is the potential due to the nuclei and $F_{HK}$ is a universal functional independent on the external potential and is given as
\begin{equation}\label{fhk}
 F_{HK}[\rho] = \big \langle\psi[\rho]|\hat{T}+\hat{V}_{ee}|\psi[\rho]\big \rangle
 \end{equation}
Given this equation, the variational principle can be utilized to extend the electron density. For an electron
system with density ($\rho$), $\rho_{o}$ and $E_{o}$ can be estimated accurately through minimization of the energy functional $E[\rho]$ $\big(ie.
E_{o} = minE[\rho (r)]\big)$. 
This makes the the electron density zero or positive at any point in space
\begin{equation}\label{zero}
\int\rho(r)dr = N, \hspace{10mm} \rho(r) \geq 0, \forall r
\end{equation}
This implies that the exact ground-state density reduces the energy functional and is applicable when $\rho$ = $\rho_{o}$ for;
\begin{equation}\label{exact}
E[\rho] = E_{o}
\end{equation}
Practically, the universal functional is not solvable, so approximation on the wave function is usually neglected and shifted towards the 
the universal functional. This is achieved through a method proposed by Kohn and Sham. 
\section{The Kohn-Sham (KS) Theorem}
The key idea behind the KS theory is the calculation of the kinetic energy of the N-electron system under the assumption that the electons are 
non-interacting \citep{kohn1965self} and considering the variable $\hat{V}_{ee}[\rho]$ in the universal functional as zero. This reduces the
$F_{HK}$ to exact kinetic energy function which is expressed as
\begin{equation}\label{ks}
T_{s}[\rho] = \sum_{i=1}^{N_{elec}}\big \langle\psi_{s}\bigg|-\frac{\nabla^{2}}{2}\bigg|\psi_{s}\big \rangle
\end{equation}
$\psi=|\psi_{1},...\psi_{N_{elec}}|$ denotes that the kinetic energy is obtained from slater determinant of N-electron system.\\
The potential that result to the correct electron density in the non-interacting reference system is still not known, however, it can be 
obtained by comparing the Euler-Lagrange equations for the two systems.  
This comparison gives the 
external potential of the non-interacting KS system, $V_{s}$ containing the effects of all electron interactions
\begin{equation}\label{vss}
V_{s}(r) = V(r) + \int\frac{\rho(r^{'})}{|r-r^{'}|}dr^{'} + V_{exc}(r)
\end{equation}
with the Euler-Lagrange equations as
\begin{equation}\label{elg1}
0 = \frac{\partial}{\partial\rho}\bigg[-\mu\bigg(\int\rho(r)dr-N\bigg) + E[\rho]\bigg] 
\end{equation}
\begin{equation}\label{elg2}
 0 = \frac{\partial}{\partial\rho}\bigg[-\mu_{s}\bigg(\int\rho(r)dr-N\bigg) + E_{s}[\rho]\bigg]
\end{equation}
Based on this the exact total energy as functional of the electron density can be constructed as
\begin{equation}\label{tenergy}
E_{s}[\rho] = T_{s}[\rho] + E_{ne}[\rho] + J[\rho] + E_{xc}[\rho] = T_{s}[\rho] + \int \rho(r)V_{s}(r)dr
\end{equation}
$T_{s}[\rho]$ is the KS kinetic energy, $E_{ne}[\rho]$ is the potential energy between the electrons and nuclei. $J[\rho]$ and $E_{xc}[\rho]$ 
are the coulomb repulsion and exchange correlation energy respectively; and are given as
\begin{equation}\label{jc}
J[\rho] = \int\frac{\rho(r)\rho(r^{'})}{|r-r^{'}|}drdr^{'} 
\end{equation}
and
\begin{equation}\label{exc}
E_{xc}[\rho] = (T[\rho]-T_{s}[\rho]) + (E_{ee}[\rho]-J[\rho])
\end{equation}
The functional derivative of the exchange correlation energy $E_{xc}[\rho]$ gives the exchange correlation potential, $V_{exc}$ as
\begin{equation}\label{vxc}
V_{exc}(r) = \frac{\partial E_{xc}[\rho]}{\partial\rho(r)}
\end{equation}
With the calculation of the electron density from the occupied KS oribtal functions obtained from
\begin{equation}\label{obf}
\rho(r) = \sum_{i=1}^{N_{elec}}\bigg|\psi_{i}(r)\bigg|^{2}
\end{equation}
The Hamiltonian of this non-interating KS system obtained from the external potential corresponding to the electron density can be given as
\begin{equation}\label{ha1}
\hat{H} = -\sum_{i=1}^{n}\frac{\nabla_{i}^{2}}{2} + \sum_{i=1}^{n}V_{s}[\rho](r_{i})
\end{equation}
Calculation of this Hamiltonian is computationally less complicated and yields the solution corresponding to the KS set of N of one electron
equations. The expectation eigenvalues and eigenvectors from this set of equations are respectively the one electron KS orbital functions, 
$\psi_{i}$ and the KS orbital energies, $\epsilon_{i}$
\begin{equation}\label{ks111}
\bigg[\bigg(-\frac{\nabla^{2}}{2} + V_{s}[\rho](r)\bigg) - \epsilon_{i}\bigg]\psi_{i}(r) = 0 
\end{equation}
\section{The Frozen Density Embedding (FDE) Method in DFT}
The main idea of the FDE formalism \citep{wesolowski1993frozen,gomes2012quantum} is the partitioning of the total electron density $\rho_{tot}(r)$ of a system
into a number of electron density subsystems. It is represented as the sum of the densities
$\rho_{\alpha}(r)$ and $\rho_{\beta}(r)$ (ie. $\rho_{tot}(r)=\rho_{\alpha}(r) +\rho_{\beta}(r)$), that correspond to the subsystem of interest and subsystem
of the environment respectively. Based on this partitioning, the total energy of the system, $E[\rho_{tot}]$, can be represented as the sum
of subsystem energies and an interaction energy
\begin{equation}\label{int}
\begin{split}
E[\rho_{\alpha}, \rho_{\beta}] =  E_{nn} + \int\rho_{tot}(r)(V_{\alpha}^{nuc}(r) + V_{\beta}^{nuc}(r))dr
+ \frac{1}{2}\int\int\frac{\rho_{tot}(r)\rho_{tot}(r^{'})}{|r-r^{'}|}drdr^{'}\\
\hspace{-10mm} + E_{xc}[\rho_{tot}]+ E_{xc}^{nadd}[\rho_{\alpha}, \rho_{\beta}]
+ T_{s}[\rho_{tot}]+ T_{s}^{nadd}[\rho_{\alpha}, \rho_{\beta}]
\end{split}
\end{equation}
where $E_{nn}$ is the nuclear repulsion energy, $V_{\alpha}^{nuc}$ and $V_{\beta}^{nuc}$ are the nuclear potential of the subsystem  $\alpha$ and $\beta$ respectively. 
$E_{xc}[\rho_{tot}]$ is the exchange-correlation energy functional, $T_{s}[\rho_{tot}]$ is the kinetic energy of the non-interating reference
system. $E_{xc}^{nadd}$ and $T_{s}^{nadd}$ are the nonadditive contributions due to exchange-correlation and kinetic energy respectively and are
defined as
\begin{equation}\label{inta}
\begin{split}
E_{xc}^{nadd}[\rho_{\alpha}, \rho_{\beta}] = E_{xc}[\rho_{\alpha}, \rho_{\beta}] - E_{xc}[\rho_{\alpha}] - E_{xc}[\rho_{\beta}] \\
T_{s}^{nadd}[\rho_{\alpha}, \rho_{\beta}] = T_{s}[\rho_{\alpha}, \rho_{\beta}] - T_{s}[\rho_{\alpha}] - T_{s}[\rho_{\beta}]
\end{split}
\end{equation}
The FDE uses only the electron density in the calculation of interaction between subsystems without the sharing of orbital information among the 
subsystems as KS approach or by optimizing the wave function of the interacting system for wave function theory (WFT) formalism, where 
the subsystem of interest is treated with wave function (ie. $\rho_{\alpha}$ = $\rho_{\alpha}^{DFT}$ is replaced 
with $\Psi_{\alpha}^{WFT}$) and the environment estimated  from a DFT calculation ($\rho_{\beta}$ = $\rho_{\beta}^{DFT}$).
This approach makes sure that calculation of exact exchange among the subsystems is ignored \citep{gomes2008calculation} and the resulting interaction energy
given as
\begin{equation}\label{intera}
\begin{split}
E^{int}[\rho_{\alpha}, \rho_{\beta}] = E_{nn}^{int} + \int\rho_{\alpha}(r)V^{\beta}_{nuc}(r)dr + \int\rho_{\beta}(r)V^{\alpha}_{nuc}(r)dr\\
+ \int\int\frac{\rho_{\alpha}(r)\rho_{\beta}(r^{'})}{|r-r^{'}|}drdr^{'} + E_{xc}^{nadd}[\rho_{\alpha}, \rho_{\beta}] + T_{s}^{nadd}[\rho_{\alpha}, \rho_{\beta}]
\end{split}
\end{equation}
where $E_{nn}^{int}$ is the interacting nuclear repulsion energy.\\
Based on the $\rho_{tot}(r)=\rho_{\alpha}(r) +\rho_{\beta}(r)$ of the system, optimization process is applied on the subsystems by considering 
the electron density $\rho_{\beta}(r)$ 
of the environment is frozen, on the assumption that the electron density of the subystem of interest, $\rho_{\alpha}(r)$ is does not interact 
with the environment. This minimization of the total energy of the system in relation to the subsystem of interest generates an 
Euler-Lagrangian equation that keeps the number of electrons in the subsystem of interest fixed \citep{hofener2012molecular}.%,hofener2013solvatochromic}.
\begin{equation}\label{inter}
\frac{\partial E_{\alpha}[\rho_{\alpha}]}{\partial\rho_{\alpha}} + \frac{\partial E_{int}[\rho_{\alpha},\rho_{\beta}]}{\partial\rho_{\alpha}} =\mu
\end{equation}
With this assumption, an effective embedding potential 
can be deduced from the interaction energy and included in the subsystems treatment of the molecular system. 
Given that the effect of $\rho_{\beta}(r)$ is represented by this embedding term in 
the potential of the $\rho_{\alpha}(r)$ subsystem, then KS-like equation in subsystem DFT can be derived and its solution can yield the 
exact ground-state electron density \citep{wesolowski1993frozen}
\begin{equation}\label{ns3}
\bigg[-\frac{\nabla^{2}}{2} + V_{eff}^{KS}[\rho_{\alpha}](r) + V_{eff}^{emb}[\rho_{\alpha}, \rho_{\beta}](r) - \epsilon_{i}\bigg]\psi_{i}^{(\alpha)}(r) = 0 
\end{equation}
where $V_{eff}^{KS}[\rho_{\alpha}](r)$ is the KS effective potential of the subsystem of interest and $V_{eff}^{emb}[\rho_{\alpha}, \rho_{\beta}]$ is 
the effective embedding potential, which can be expressed as 
\begin{equation}\label{ns4}
\begin{split}
V_{eff}^{emb}[\rho_{\alpha}, \rho_{\beta}](r) =  V_{nn}^{\beta}(r) + \int\frac{\rho_{\beta}(r^{'})}{|r-r^{'}|}dr^{'} + \frac{\partial E_{xc}[\rho]}{\partial\rho}\bigg|_{\rho=\rho_{tot}}\\
-\frac{\partial E_{xc}[\rho]}{\partial\rho}\bigg|_{\rho=\rho_{\alpha}} + \frac{\partial T_{s}[\rho]}{\partial\rho}\bigg|_{\rho=\rho_{tot}} + \frac{\partial T_{s}[\rho]}{\partial\rho}\bigg|_{\rho=\rho_{\alpha}}
\end{split}
\end{equation}
\section{Coupled-Cluster (CC) Theory}
The CC method employs exponential expansion of a wave function to approximately find the 
ground-state eigenvalue and its corresponding eigenvalue of the Schrodinger equation. 
Given the CC wave function ansatz \citep{bartlett2007coupled} 
 with the cluster operator,  $\hat{T}$
\begin{equation}\label{ns5}
|\psi_{CC}\big \rangle  = exp(\hat{T})|\phi_{HF}\big \rangle 
\end{equation}
\begin{equation}\label{ns6}
\hat{T} = \sum_{i=1}^{n}\sum_{\mu_{i}}t_{\mu_{i}}\hat{\tau}_{\mu_{i}}
\end{equation}
where the state $|\phi_{HF}\big \rangle $ is the Hartree-Fock reference wave function, $t_{\mu_{i}}$ as the excitation amplitude and $\hat{\tau}_{\mu_{i}}$ as 
the i-electron excitation operators. The introduction of the CC wave function ansatz into the time-independent Schrödinger equation and projecting it onto the HF reference state
yields the CC energy, $E_{cc}$ and projection onto the excitations of the 
reference state gives the CC amplitudes, $e_{\mu_{i}}$. The CC energy and 
amplitudes may be expressed as
\begin{equation}\label{ns7}
E_{cc}=\big \langle \phi_{HF}|\hat{H}exp(\hat{T})|\phi_{HF}\big \rangle 
\end{equation}
\begin{equation}\label{aede1}
e_{\mu_{i}} = \big \langle \mu_{i}|exp(-\hat{T})\hat{H}exp(\hat{T})|\phi_{HF}\big \rangle  =0
\end{equation}
where $\big \langle \mu_{i}|$ $=$ $\big \langle \phi_{HF}|\hat{\tau}^{\dagger}_{\mu_{i}}, \big \langle \mu_{i}|v_{j}\big \rangle  =\partial_{\mu,v}\partial_{i,j}$\\
This excitation manifold is uniquely defined with respect to the Fermi vacuum in the sense
that it contains annihilators from the occupied orbitals and creators from the unoccupied 
orbitals of the reference determinant $|\phi_{HF}\big \rangle $ \citep{koch1990coupled,hanrath2005exponential}. 
The truncation of the expansion of the cluster operator gives rise to
a series of approximate models that introduce single, double and triple excitations into the wave function \citep{bartlett2007coupled}. The
electronic parameters obtained from this CC approach are 
not variational and may lead to inaccurate determination of the  CI 
energy. In this respect, it is appropriate to introduce a configuration-interaction Lagrangian
which is variational and optimizing the energy function with respect to all variational parameters, $v_{j}$  
\citep{kongsted2003coupled}
\begin{equation}\label{aede3}
0 = \frac{\partial L_{cc}(t,\bar{t})}{\partial\bar{t}_{\mu_{i}}} = e_{\mu_{i}}(t,\bar{t})
\end{equation}
\begin{equation}\label{aede4}
\frac{\partial L_{cc}(t,\bar{t})}{\partial t_{v_{j}}} = \frac{\partial E_{cc}(t)}{\partial t_{v_{j}}} + \sum_{i,\mu_{i}}\bar{t}_{\mu_{i}}\frac{\partial e_{\mu_{i}}(t)}{\partial t_{v_{j}}}
\end{equation}
This gives the configuration interaction Lagrangian
\begin{equation}\label{aede5}
L_{cc}(t,\bar{t}) = E_{cc}(t)+ \sum_{i,\mu_{i}}\bar{t}_{\mu_{i}} 
\frac{\partial e_{\mu_{i}}(t)}{\partial t_{v_{j}}} = \big \langle \Lambda|[\hat{H}, \hat{\tau}_{v_{j}}]|\psi_{CC}\big \rangle 
\end{equation}
where vector $\bar{t}$ contains the Lagrangian multipliers, $\big \langle \Lambda|$ is the bra state and the normal coupled-cluster wave function at the ket state and can be defined as
\begin{equation}\label{aede6}
\big \langle \Lambda| = \big(\big \langle \phi_{HF}|+\sum_{i,\mu_{i}}\bar{t}_{\mu_{i}}\big \langle \mu_{i}|\big)e^{-\hat{T}}
\end{equation}
\section{Coupled-Cluster in DFT (CC-in-DFT)}
The equation of the CC-in-DFT can be derived through the expansion of the interaction term as a function of the 
coupled-cluster amplitude and the Lagrangian multipliers as
\begin{equation}\label{aede2}
E_{int}^{cc} = E_{int}^{[o]}[\rho_{cc}(t,\bar{t}),\rho_{DFT}(k_{II})] + E_{int,t}^{[1]}t + E_{int,\bar{t}}^{[1]}\bar{t} + 
E_{int,k_{II}}^{[1]}k_{II} + ....
\end{equation}
The coupled-cluster electron density $\rho_{cc}(t,\bar{t})$ is constructed as an expectation value from the bra state 
and the normal coupled-cluster wave function at the ket state $\big(\big \langle \Lambda|\big)$ 
so that the time variation of the 
norm of the bra-ket is preserved \citep{koch1990coupled,hofener2012molecular}.
\begin{equation}\label{aede2}
\big \langle \Lambda(t)|CC(t)\big \rangle  = 1
\end{equation}
and is given as 
\begin{equation}\label{aede2}
\rho(r,t) = \big \langle \Lambda(t)|\hat{\rho}(r)|CC(t)\big \rangle  \sum_{pq}\psi_{p}(r)\psi_{q}(r)\hat{D}_{pq}(t)
\end{equation}
where $\hat{D}$ is the CC density matrix for one-electron.\\
The first perturbation on the CC electron density to generate the zeroth-order Lagrangian \citep{kongsted2003coupled},  
result in the estimation of the total energy of the CC-in-DFT, which is expressed as
\begin{equation}\label{aede2}
L^{[o]} =\big \langle \phi_{HF}|\hat{H}_{\mp}|\phi_{HF}\big \rangle  + E_{cc}^{[o]} + E_{int}^{[o]} + E_{DFT}^{[o]}
\end{equation}






\bibliography{the.bib}
\bibliographystyle{apalike}
\end{document}
