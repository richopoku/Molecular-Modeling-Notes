\documentclass[a4paper,11pt]{report}
\usepackage[utf8]{inputenc}
\usepackage{graphicx}
\usepackage{float}
\usepackage{amsmath}
\usepackage{amssymb}
\usepackage{natbib}
\graphicspath{{images/}}
\usepackage[width=150mm,top=15mm,bottom=25mm]{geometry}
\usepackage{times}
\usepackage{textcomp}
\usepackage{caption}
\usepackage{booktabs}
\renewcommand{\baselinestretch}{2}
\renewcommand{\figurename}{\textbf{Figure}}
\renewcommand{\contentsname}{Table of Contents}
\renewcommand{\bibname}{References}
\renewcommand{\baselinestretch}{2}
\renewcommand\bibname{References}


\thispagestyle{empty}



\begin{document}
%\maketitle
%\chapter*{}


\pagenumbering{roman}

%\chapter*{}

%\chapter*{}

%\chapter*{}

%\tableofcontents
%\listoffigures
%\listoftables
%\chapter{Introduction}
\pagenumbering{arabic}

\chapter{CORE SPECTROSCOPY}
X-ray photoelectron spectroscopy (XPS) is one of the most well-established methods to provide detailed
information about the chemical state and electronic properties of molecules given its surface sensitivity.
XPS is based on the interaction of X-ray with molecules to cause excitation of the core electrons out of their
bound orbits. The kinetic energies and angular distributions from the excitation of the core electrons obtained 
gives an insight into the structure of the molecules. 
The X-ray induces peturbation in the core electrons leading to transition in which electrons from 
a bonding orbital can be moved to anti-bonding orbital with core ionization. Following the photoelectron emission,
unpaired electrons may couple with other unpaired electrons, resulting in ion with several possible final states with 
different energies (multiplet splitting). That is, the energy generated as a results of the relaxation of the final 
state configuration due to loss of sreening effect of the core electron level exites electrons in valence level to 
unbound states. 
The difference between the photon energy of the incident x-ray(hv), the element and chemical state  binding energy 
($E_{bind}$) of the core electron, and a workfunction dependent ($\theta_{f}$) gives the kinetic energy ($E_{kin}$) 
of the photoelectron, which is expressed as
\begin{equation}\label{corecomb}
E_{kin} = hv - E_{bind} - \theta_{f}
\end{equation}
The $E_{bind}$ of the core electron is proportional to the $E_{kin}$ for the detected electron.
\begin{figure}[H]\large
\includegraphics[width = 9.0cm, trim = 0cm 0cm 9cm 0cm]{corecomb.png}
\caption{}
\label{d2}
\end{figure}
The XPS gives the negative Hartree-Fock orbital energy (eigenvalues) based on the assumption that
the occupied orbitals from the Hartree-Fock calculations (initial state) is equal to the ionization energy
to the ion state (final state) formed by the removal of electron from the orbital provided the distributions of the 
remaining electrons do not change (frozen) The energies of the orbital give the amount of energy capable to remove (ionize) 
the electron out of the molecular orbital, which corresponds to a negative ionization potential. In this respect, the ionization
energies are directly related to the energies of molecular orbitals and the Koopman's binding energy is given as
\begin{equation}\label{koopcomb}
I_{j} = -\epsilon_{j}, \hspace{10mm} E_{B,K} = -\epsilon_{B,K}
\end{equation}
\begin{figure}[H]\large
\includegraphics[width = 8.0cm, trim = 0cm 0cm 9cm 0cm]{koopcomb.png}
\caption{}
\label{d3}
\end{figure}

\section{XPS studies on complex systems}
Intensity of the ionic Cl intensity depletes at the uppermost surface compared to the deeper region, followed by a slower intensity decay
with depth than HCl, which an indication that HCl is only formed at the surface,  while ionic Cl resides deeper in the ice surface region.
The presence of ionic Cl induces significant changes to the hydrogen bonding network of water
ice by binding the molecules into solvation (perturbation of the hydrogen 
bonding network by ionic Cl). The  water  molecules  within  the  interfacial
region  are mobile or flexible enough to accommodate the need of ionic 
Cl. HCl leads to an extended disorderd layer on the ice.
Significant fraction of water molecules are engaged in solvating the ionic
Cl and is similar to those in aqueous solution.
Chemical shift in binding energy of 2.2 eV between covalent and ionic Cl 
was observed for both Cl 1s singlets and Cl 2p doublets and 
were determined by the flexibility in the hydrogen bonding network.
Full dissociation of HCl on ice surface was observed at few nanometers. The coexistence of both
molecular HCl and ionic Cl exhibited Janus type behavior of the acid which is
different from liquid bulk phase and the dissociation occurs only at some distinct 
location deeper within the air-ice interface.  The increasing presence of HCl
leads to the line shape of the NEXAFS to transits from solid ice  towards  
that  of  liquid  water between the 
main-edge peak and the post-edge peak.  HCl partially  dissociates  upon  
adsorption  at  a fraction of a Langmuir monolayer at 253K. A sharp  decrease
of  the  apparent  HCl/$Cl^{-}$intensity  ratio  in  individual  XPS  spectra  with
increasing probing depth, indicating that the presence of molecular HCl
relative to Cl-is strongly favored at the outermost surface of
ice. Their model indicates that the presence of molecular HCl is limited
to a surface layer less than observation of molecular HCl at 
warm ice surface. HCl adsorption on ice at 253 K deviates from 
the Langmuir kinetics \citep{kong2017coexistence}

HCl dissociate into ionic Cl and $H^{+}$ by making 3 hydrogen bonds by direct
reaction with water ice and by self-solvation at low temperatures. Chemical shift in binding energy of 2.6 eV and 1.3 eV were 
observed between covalent Cl and and ionic Cl 1s singlet and 2P doublet respectively. The disapperance of $\sigma$(H-Cl) ressonance (transition
from HOMO to LUMO) and the similarity of the XPS spectrum to that of atomic chlorine indicated the dissociation of H-Cl upon adsorption on ice.
The formation of the Cl anion pulling the ionization potential (IP) downward led to the 4s Rydberg transition is also shifted from 203.3 eV
to 200.0 eV. HCl is fully ionized by making three hydrogen bonds with water 
(contact ion pair $H_{3}O^{+}$:$Cl^{-}$) or HCl is fully ionized by making three hydrogen
bonds, two with water and one with a solvating HCl (that is not dissociated).
HCl is adsorbed molecular, singly (HCl[1]) or doubly (HCl[2]) hydrogen 
bonded with water \citep{parent2011hcl}. 


The C 1s spectra of acetone adsorption on ice showed a chemical shift binding
energy of 3.0 eV which were assigned to methyl and carnonyl group. The 
O K-edge showed transition of O 1s core level into unoccupied $\pi^{*}$ orbital.
The carbonyl oxygen formed a strong hydrogen bond with a hydrogen atom of a dangling OH on the
ice, while one or two weak bonds are involve in the hydrogen atoms of methyl
groups and oxygen atoms on the ice surface. Actone does not induce premelting 
transition at the ice surface at temperatures up to 245 K. Acetone adsorbs with
its molecularplane almost parallel to the ice surface. There is a weak interaction 
between acetone and ice \citep{starr2011acetone}.

The C 1s XPS spectra of 2-propanol adsorption on ice surface at 227 K revealed 
a chemical  shift of binding enrgy of 1.5 eV, which were assigned to alcohol 
group and 2 methyl group. 2-propanol does not interact with ice surface strongly 
. There is minimal lateral interactions between adsorbed 2-propanol molecules and 
stong hydrogen bonding of the alcohol OH group with ice surface.
Ice–OH interaction dominates the overall adsorption energetics which is 
consistent with a Langmuirian mechanism where lateral interactions are
negligible \citep{newberg2015adsorption}.

The C 1s XPS  spectra displayed two features witha chemical shift of binding energy 
of 3.8 eV, which were assigned to protonated and deprotonated acetic acid. 
The acetic acid resides within the topmost bilayers of the ice surface. 
The O-edge spectra indicated only minor perturbations of the hydrogen bonding 
network and that acetic acid does not lead to extended disordered layer on the ice 
surface between 230 K and 240 K. The O K-edge NEXAFS are sensitive to the local 
hydrogen-bonding structure and can help asses whether acid adsorption has an 
effect on the extent of the QLL or surface premelting. The degree of protonation
of acetic acid at the ice surface, about 60\%, which is higher than that
of an aqueous solution \citep{krepelova2013adsorption}.

\section{Gas phase studies with XPS and theoretical claculations}
Photoelectron spectroscopy and density functional theory were used to study
Uraniumpentahalides, of particular interest is Uraniumpentachloride complexes.
The PES spectra revealed that the ionic pentachloride complex is electronically 
stable at a adiabatic binding energy of 4.76 eV at 157nm, which is also an electron 
affinity of neutral pentachloride. Dissociation of the pentachloride anion 
was observed by the PES at 245nm. The DFT approach and CC methods calculations 
(in particular the WFT approach and the EOM-CCSD(T) on the 
uranium pentchloride complexes provided reliabe ground- and excitation-state 
energies. The chemical bonding of the U-X interactions were dominated in the 
pentachloride anion were dominated by ionic bonding. 
Kohn-Sham orbitals and their orbital energies are significant in the interpretation of electron detachments 
and spectroscopy properties. The photoelectron spectroscopy and the 
electronic structure calculations provided detailed information  in the valence region n
of the molecular orbitals of the pentachloride complexes. The U-X bond
strength decreases from F to I, as a result of the reduced electrostatic
interactions, amid the increase of the relatively weak covalent
interactions. The theoretical investigations revealed that the 
ground-state of the pentachloride anion is a triplet\citep{su2013joint}


The PES spectra of the two hydrated species were similar to that of the
solute $UO_{2}F_{4}^{2-}$ dianion, except a systematic shift of the electron binding
energies due to the solvation effects. The binding energy of the solute 
$UO_{2}F_{4}^{2-}$ dianion and the hydrated $UO_{2}F_{4}^{2-}$ dianion were 2.0 eV and 1.7 eV 
respectively. 
The adiabatic detachment energy (ADE) and vertical detachment energy (VDE)
calculated with the scalar relativistic (SR) CCSD(T) yields accurate results,
which is in agreement with experiment
The use of spin-orbit (SO) corrections from the PBE calculations and
the SR CCSD(T) gives ADE and VDE of $UO_{2}F_{4}^{2-}$ to be
around 1.20 and 1.72 eV respectively, which are cmparable  
to the experiment but a little too high.
The partial inclusion of multi-reference treatment for excited states 
in the CCSD(T) yieids accurate energies
Both DFT and CCSD(T) calculations proved capable by complementing each
other in the accurate 
description of coulomb repulsion in the dianions and the U-O chemical 
bonding. DFT and CCSD(T) yielded accurate ground state transitions that 
are consistent with experimental results \citep{dau2012observation}.

\section{DFT}
Kohn-Sham DFT-based molecular dynamics provide accurate description of strong acids interaction with water molecules. The DFT-calculations 
indicate that $Cl^{-}$ is solvated with approximately six water molecules in symmetric hydration shell; whereas the molecular HCl only donate 
one strong hydrogen bond and does not accept any. The free-energy separating the undissociated form of the acids from the ionic form depend 
on the identity of the acid \citep{baer2014investigation}.

DFT provide accurate description of hydrogen bonding between hydrogen halides and water molecules comparbable to ab inito with less computational 
cost. Hydrogen halides dissociation is governed by dipole interactions as the dipoles in the water molecules stabilize the ion-pair formed upon 
dissociation \citep{cabaleiro2002computational}.

DFT and coupled-cluster theory with singles and doubles excitation gives accurate binding energies and electronic properties 
of hydrogen halides interaction with water molecules comparble to ab inito method \citep{odde2004dissociation}

DFT with GGA provides accurate description of the nature of bond with less computational hydrogen halides interaction with water molecules. The 
chemical shift in binding energies at the core-level indicates the existence of molecular and ionic Cl. The study proved to be accurate in 
describing the bond breaking and dynamics of the system in self-consistent manner \citep{calatayud2003ionization}.

Studies on the structure and binding energies of ion-$\pi$ complexes using DFT/CSSD and ab inito determined that DFT describes ion-$\pi$ 
interaction accurately comparable to ab inito method at less computational cost. DFT neglects dispersion effects and overestimate ion-$\pi$ 
interactions making it essential in the study of complex systems \citep{quinonero2005structure}

\section{QM/MM approaches}
Recent studies based QM/MM indicate that classical force based on parametrization of potential energy surface applied on chemical reactions
which are strongly influenced by the environment such as the dissociation of HCl at ice surfaces are prone to problems of potential 
transferability and provide poor description of intermolecular interactions and  hydrogen bonding. It underestimates the proton 
affinity of the chloride anion and overestimate the proton transfer barrier. semi-empirical approaches need reoptimization of the core-core 
functions and reparameterization of the electronic terms in order to provide accurate description of interaction energies. 
Chemical shift in the binding energies between the ionic and the covalent was not resolved \citep{arillo2007can}.

Model parameterized to ab initio calculations determined the bindinding energies of hydrogen halides in solvent. The study indicates that 
semi-empirical approach need parameterization of electronic functions to provide accurate description of intermolecular interactions since
it was difficult to sample transitions from the molecular form to the contact ion pair form, thereby leading to neglection of the molecular form of the hydrogen 
halides in the potential of mean force \citep{wick2017comparison}.



\chapter{DFT and MBPT}
The of study of electron affinity of water determined that density functional calculations and G0W0 approximation in the many-body 
perturbation theory (MBPT) is the most accurate method for computing the interatomic interactions  and binding energies of electrons
in water in gaseous and condensed phases. However, the G0W0 calculations yielded a mean absolute error (MAE) with respect to measured
spectra of 0.18 eV for trajectories including nuclear quantum effects(NQE) and 0.38 eV for classical trajectories. The electron affinity
computed with coupled cluster with single, double, and perturbative triple excitations [CCSD(T)] method indicate that the CCSD(T) agrees
with, G0W0 for quasiparticle energies agree within 0.05 eV \citep{gaiduk2018electron}.

The Green function (GF) method like GW approximation and CC approach are the most suitable approach up to now for studying the excited-state
properties of molecular systems. The GW approximation on the other hand is considered to be the the most fruitful approximation for self-energy
calculations. However, despite the success of the GW approximation in self-energy calculations, it lacks the ability to account for the
presence of spin-dependence in the coulombic interactions, not self-consistent and computationally costly and inaccurate in its application
to large and complex systems \citep{aryasetiawan1998gw}.




\chapter{Relativistic coupled-cluster}

\chapter{Density functional theory}


\chapter{Frozen density embedding}
For FDE to be able acccount for polarization of the solvent, there is the need to relax the solvent density with respect to the solute
in freeze-and-thaw cycles. FDE presents a exact framework that can predict accurately the ground state properties (dipole and
quadrupole moments) of short range effects
such direct hydrogen bonding between solute and solvent as well Pauli repulsion of the solvent  on diffuse diffuse excited states.
FDE provides accurately excitation energies comparable to supermolecular calculations of solvents \citep{jacob2006comparison}.

FDE presents an exact framework that allows DFT to be substituted with WFT (WFT-in-DFT). This WFT-in-DFT approach allows the incorporation 
of coupled-cluster density that makes it possible to calculate the embedding potential for the ground state as well as the excited states via the
calculation of the kernel contributions to the CC Jacobian matrix. WFT-in-DFT provides accurate variation in electronic spectra induced by solvent
molecules. This approach allows the efficient calculation of excitation energies of solvents without compromising the accuracy \citep{hofener2013solvatochromic}.





\bibliography{con2.bib}
\bibliographystyle{apalike}
\end{document}

=======
\documentclass[a4paper,11pt]{report}
\usepackage[utf8]{inputenc}
\usepackage{graphicx}
\usepackage{float}
\usepackage{amsmath}
\usepackage{amssymb}
\usepackage{natbib}
\graphicspath{{images/}}
\usepackage[width=150mm,top=15mm,bottom=25mm]{geometry}
\usepackage{times}
\usepackage{textcomp}
\usepackage{caption}
\usepackage{booktabs}
\renewcommand{\baselinestretch}{2}
\renewcommand{\figurename}{\textbf{Figure}}
\renewcommand{\contentsname}{Table of Contents}
\renewcommand{\bibname}{References}
\renewcommand{\baselinestretch}{2}
\renewcommand\bibname{References}


\thispagestyle{empty}



\begin{document}
%\maketitle
%\chapter*{}


\pagenumbering{roman}

%\chapter*{}

%\chapter*{}

%\chapter*{}

%\tableofcontents
%\listoffigures
%\listoftables
%\chapter{Introduction}
\pagenumbering{arabic}

\chapter{CORE SPECTROSCOPY}
X-ray photoelectron spectroscopy (XPS) is one of the most well-established methods to provide detailed
information about the chemical state and electronic properties of molecules given its surface sensitivity.
XPS is based on the interaction of X-ray with molecules to cause excitation of the core electrons out of their
bound orbits. The kinetic energies and angular distributions from the excitation of the core electrons obtained 
gives an insight into the structure of the molecules. 
The X-ray induces peturbation in the core electrons leading to transition in which electrons from 
a bonding orbital can be moved to anti-bonding orbital with core ionization. Following the photoelectron emission,
unpaired electrons may couple with other unpaired electrons, resulting in ion with several possible final states with 
different energies (multiplet splitting). That is, the energy generated as a results of the relaxation of the final 
state configuration due to loss of sreening effect of the core electron level exites electrons in valence level to 
unbound states. 
The difference between the photon energy of the incident x-ray(hv), the element and chemical state  binding energy 
($E_{bind}$) of the core electron, and a workfunction dependent ($\theta_{f}$) gives the kinetic energy ($E_{kin}$) 
of the photoelectron, which is expressed as
\begin{equation}\label{corecomb}
E_{kin} = hv - E_{bind} - \theta_{f}
\end{equation}
The $E_{bind}$ of the core electron is proportional to the $E_{kin}$ for the detected electron.
\begin{figure}[H]\large
\includegraphics[width = 9.0cm, trim = 0cm 0cm 9cm 0cm]{corecomb.png}
\caption{}
\label{d2}
\end{figure}
The XPS gives the negative Hartree-Fock orbital energy (eigenvalues) based on the assumption that
the occupied orbitals from the Hartree-Fock calculations (initial state) is equal to the ionization energy
to the ion state (final state) formed by the removal of electron from the orbital provided the distributions of the 
remaining electrons do not change (frozen) The energies of the orbital give the amount of energy capable to remove (ionize) 
the electron out of the molecular orbital, which corresponds to a negative ionization potential. In this respect, the ionization
energies are directly related to the energies of molecular orbitals and the Koopman's binding energy is given as
\begin{equation}\label{koopcomb}
I_{j} = -\epsilon_{j}, \hspace{10mm} E_{B,K} = -\epsilon_{B,K}
\end{equation}
\begin{figure}[H]\large
\includegraphics[width = 8.0cm, trim = 0cm 0cm 9cm 0cm]{koopcomb.png}
\caption{}
\label{d3}
\end{figure}

\section{XPS studies on complex systems}
Intensity of the ionic Cl intensity depletes at the uppermost surface compared
to the deeper region, followed by a slower intensity decay with depth than HCl, 
which an indication that HCl is only formed at the surface,  while ionic Cl 
resides deeper in the ice surface region. The presence of
ionic Cl induces significant changes to the hydrogen bonding network of water
ice by binding the molecules into solvation (perturbation of the hydrogen 
bonding network by ionic Cl). The  water  molecules  within  the  interfacial
region  are mobile or flexible enough to accommodate the need of ionic 
Cl. HCl leads to an extended disorderd layer on the ice.
Significant fraction of water molecules are engaged in solvating the ionic
Cl and is similar to those in aqueous solution.
Chemical shift in binding energy of 2.2 eV between covalent and ionic Cl 
was observed for both Cl 1s singlets and Cl 2p doublets and 
were determined by the flexibility in the hydrogen bonding network.
Full dissociation of HCl on ice surface was observed at few nanometers. The coexistence of both
molecular HCl and ionic Cl exhibited Janus type behavior of the acid which is
different from liquid bulk phase and the dissociation occurs only at some distinct 
location deeper within the air-ice interface.  The increasing presence of HCl
leads to the line shape of the NEXAFS to transits from solid ice  towards  
that  of  liquid  water between the 
main-edge peak and the post-edge peak.  HCl partially  dissociates  upon  
adsorption  at  a fraction of a Langmuir monolayer at 253K. A sharp  decrease
of  the  apparent  HCl/$Cl^{-}$intensity  ratio  in  individual  XPS  spectra  with
increasing probing depth, indicating that the presence of molecular HCl
relative to Cl-is strongly favored at the outermost surface of
ice. Their model indicates that the presence of molecular HCl is limited
to a surface layer less than observation of molecular HCl at 
warm ice surface. HCl adsorption on ice at 253 K deviates from 
the Langmuir kinetics \citep{kong2017coexistence}

HCl dissociate into ionic Cl and $H^{+}$ by making 3 hydrogen bonds by direct
reaction with water ice and by self-solvation at low temperatures. Chemical shift in binding energy of 2.6 eV and 1.3 eV were 
observed between covalent Cl and and ionic Cl 1s singlet and 2P doublet respectively. The disapperance of $\sigma$(H-Cl) ressonance (transition
from HOMO to LUMO) and the similarity of the XPS spectrum to that of atomic chlorine indicated the dissociation of H-Cl upon adsorption on ice.
The formation of the Cl anion pulling the ionization potential (IP) downward led to the 4s Rydberg transition is also shifted from 203.3 eV
to 200.0 eV. HCl is fully ionized by making three hydrogen bonds with water 
(contact ion pair $H_{3}O^{+}$:$Cl^{-}$) or HCl is fully ionized by making three hydrogen
bonds, two with water and one with a solvating HCl (that is not dissociated).
HCl is adsorbed molecular, singly (HCl[1]) or doubly (HCl[2]) hydrogen 
bonded with water \citep{parent2011hcl}. 


The C 1s spectra of acetone adsorption on ice showed a chemical shift binding
energy of 3.0 eV which were assigned to methyl and carnonyl group. The 
O K-edge showed transition of O 1s core level into unoccupied $\pi^{*}$ orbital.
The carbonyl oxygen formed a strong hydrogen bond with a hydrogen atom of a dangling OH on the
ice, while one or two weak bonds are involve in the hydrogen atoms of methyl
groups and oxygen atoms on the ice surface. Actone does not induce premelting 
transition at the ice surface at temperatures up to 245 K. Acetone adsorbs with
its molecularplane almost parallel to the ice surface. There is a weak interaction 
between acetone and ice \citep{starr2011acetone}.

The C 1s XPS spectra of 2-propanol adsorption on ice surface at 227 K revealed 
a chemical  shift of binding enrgy of 1.5 eV, which were assigned to alcohol 
group and 2 methyl group. 2-propanol does not interact with ice surface strongly 
. There is minimal lateral interactions between adsorbed 2-propanol molecules and 
stong hydrogen bonding of the alcohol OH group with ice surface.
Ice–OH interaction dominates the overall adsorption energetics which is 
consistent with a Langmuirian mechanism where lateral interactions are
negligible \citep{newberg2015adsorption}.

The C 1s XPS  spectra displayed two features witha chemical shift of binding energy 
of 3.8 eV, which were assigned to protonated and deprotonated acetic acid. 
The acetic acid resides within the topmost bilayers of the ice surface. 
The O-edge spectra indicated only minor perturbations of the hydrogen bonding 
network and that acetic acid does not lead to extended disordered layer on the ice 
surface between 230 K and 240 K. The O K-edge NEXAFS are sensitive to the local 
hydrogen-bonding structure and can help asses whether acid adsorption has an 
effect on the extent of the QLL or surface premelting. The degree of protonation
of acetic acid at the ice surface, about 60\%, which is higher than that
of an aqueous solution \citep{krepelova2013adsorption}.

\section{Gas phase studies with XPS and theoretical claculations}
Photoelectron spectroscopy and density functional theory were used to study
Uraniumpentahalides, of particular interest is Uraniumpentachloride complexes.
The PES spectra revealed that the ionic pentachloride complex is electronically 
stable at a adiabatic binding energy of 4.76 eV at 157nm, which is also an electron 
affinity of neutral pentachloride. Dissociation of the pentachloride anion 
was observed by the PES at 245nm. The DFT approach and CC methods calculations 
(in particular the WFT approach and the EOM-CCSD(T) on the 
uranium pentchloride complexes provided reliabe ground- and excitation-state 
energies. The chemical bonding of the U-X interactions were dominated in the 
pentachloride anion were dominated by ionic bonding. 
Kohn-Sham orbitals and their orbital energies are significant in the interpretation of electron detachments 
and spectroscopy properties. The photoelectron spectroscopy and the 
electronic structure calculations provided detailed information  in the valence region n
of the molecular orbitals of the pentachloride complexes. The U-X bond
strength decreases from F to I, as a result of the reduced electrostatic
interactions, amid the increase of the relatively weak covalent
interactions. The theoretical investigations revealed that the 
ground-state of the pentachloride anion is a triplet\citep{su2013joint}


The PES spectra of the two hydrated species were similar to that of the
solute UO2F42- dianion, except a systematic shift of the electron binding
energies due to the solvation effects. The binding energy of the solute 
$UO_{2}F_{4}^{2-}$ dianion and the hydrated $UO_{2}F_{4}^{2-}$ dianion were 2.0 eV and 1.7 eV 
respectively. 
The adiabatic detachment energy (ADE) and vertical detachment energy (VDE)
claculated with the scalar relativistic (SR) CCSD(T) yields accurate results,
which is in agreement with experiment
The use of spin-orbit (SO) corrections from the PBE calculations and
the SR CCSD(T) gives ADE and VDE of $UO_{2}F_{4}^{2-}$ to be
around 1.20 and 1.72 eV respectively, which are cmparable  
to the experiment but a little too high.
The partial inclusion of multi-reference treatment for excited states 
in the CCSD(T) yieids accurate energies
Both DFT and CCSD(T) calculations proved capable by complementing each
other in the accurate 
description of coulomb repulsion in the dianions and the U-O chemical 
bonding. DFT and CCSD(T) yielded accurate ground state transitions that 
are consistent with experimental results.

\chapter{Relativistic coupled-cluster}

\chapter{Density functional theory}


\chapter{Frozen density embedding}





\bibliography{con2.bib}
\bibliographystyle{apalike}
\end{document}
>>>>>>> 03efdc2fc31f44ec7fc7190761aeb12a60409c67
