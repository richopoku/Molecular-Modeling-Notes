\documentclass[a4paper,11pt]{report}
\usepackage[utf8]{inputenc}
\usepackage{graphicx}
\usepackage{float}
\usepackage{amsmath}
\usepackage{amssymb}
\usepackage{natbib}
\graphicspath{{images/}}
\usepackage[width=150mm,top=15mm,bottom=25mm]{geometry}
\usepackage{times}
\usepackage{textcomp}
\usepackage{caption}
\usepackage{booktabs}
\renewcommand{\baselinestretch}{2}
\renewcommand{\figurename}{\textbf{Figure}}
\renewcommand{\contentsname}{Table of Contents}
\renewcommand{\bibname}{References}
\renewcommand{\baselinestretch}{2}
\renewcommand\bibname{References}


\thispagestyle{empty}



\begin{document}
%\maketitle
%\chapter*{}


\pagenumbering{roman}

%\chapter*{}

%\chapter*{}

%\chapter*{}

%\tableofcontents
%\listoffigures
%\listoftables
%\chapter{Introduction}
\pagenumbering{arabic}
\chapter{Summary}
The interface between air and ice governs many phenomena that are essential to chemical processes in the atmosphere.
Ice ever-present in the atmosphere and possess a wide range of microenvironments in which trace gases can interact, 
including a quasi-liquid disordered layer \citep{toubin2001dynamics, bartels2014review}. Studies connected to ice indicate that it  
can initiate and influence photochemical processes that could lead to ozone destruction \citep{solomon1999stratospheric,abbatt2003interactions}
in the troposphere. In view of this, chemical processes at the air/liquid interface have attracted a lot of scientific research,
of particular interest is acids with liquid water or water-like layer on ice.
A number of surface sensitive  techniques including Infrared spectroscopy \citep{darvas2012adsorption} and vibrational sum frequency spectroscopy \citep{gordon2018model} have been conducted over the years to derive information  about the electronic properties of such reaction through the study
of the hydrogen-bonding and proton transfer of such reaction. The reason being that the dissociation of trace gases upon interaction with
liquid water or the liquid-like layer on ice surface is related to the distortion of the hydrogen-bonding network.
X-ray photoelectron spectroscopy (XPS) is one of the most well-established methods to provide detailed information about the chemical
composition and electronic properties of molecular systems \citep{hufner2013photoelectron} given its surface sensitivity.
XPS has been applied to investigate the properties of various molecules at the core-valence region. Studies at interface between metal oxides 
such as Bismuth Vanadate \citep{starr2017combined}, Magnesium oxide \citep{newberg2011formation}, and $\alpha-Fe_{2}O_{3}$  \citep{yamamoto2010water} and aqueous electrolyte solution have indicated that
XPS is a powerful tool to investigate the binding energies at the core-valence region. These studies revealed that XPS is able to resolve
metal oxide, hydroxyl, surface molecular water and water vapor peaks arising from the core-valence region of water adsorption on metal surface 
based on their binding energies. Studies into the adiabatic electron binding energies of gaseous species including Uraniumpentachloride
\citep{su2013joint} and Uranyl tetrafluoride \citep{dau2012observation} indicate that XPS suitable to study ionic bonding of heavier halogen complexes at the core-valence
level. Studies conducted by \citep{olivieri2016quantitative} with XPS to investigate the ionization energies of aqueous solutions at the core level revealed huge
shift in ionization energies of gas phase water in the presence of liquid water and that the ionization energy of the liquid water depends 
on the chemical composition of the solution. Quite recently, the use of XPS to study the surface of saturated
sodium chloride \citep{tissot2015cation,gaiduk2016photoelectron} indicate that XPS is able to resolve ionic chlorine form aqueous solution of sodium chloride based on the core level 
shift of binding energies with respect to the bulk liquid.
XPS has also been utilized to investigate the the impact of acids on liquid water or water ice at the air-liquid interface at core-valence level. 
Investigations into the bound state of weak acid such as 2-propanol \citep{newberg2015adsorption}, acetone \citep{starr2011acetone} and acetic
acid \citep{krepelova2013adsorption} on ice surface indicte that such species do not produce any significant changes in the hydrogen-bonding
network of water ice. It has been suggested that, strong acid on ice surface on the other hand fully dissociate leading to modification
in the hydrogen-bonding network of water ice at low temperatures \citep{parent2011hcl}. Quite recently, experiment on the 
interaction of HCl at warm ice surface revealed a huge chemical shift in binding energy of 2.2 eV between covalent and ionic form of 
solvated HCl. The photoelectron spectroscopy experiment further revealed that the ionic form of the HCl perturbs the 
hydrogen bonding network of the water ice by binding them into solvation \citep{kong2017coexistence}. In this respect, there is the need for
molecular dynamics simulation in order to either provide theoretical support or give detailed information on the type of bonding at the 
core-valence regions to the experimental findings. 
Theoretical studies based on semi-empirical method aimed to provide understanding of the electronic properties of solvated HCl  
determined no chemical shift in the binding energies between ionic and covalent chlorine. These observations 
fact that semi-empirical methods overestimate the proton transfer barrier and underestimate the proton affinity of chloride anions
\citep{arillo2007can}. Recent studies probing the ground-state properties of hydrogen chloride in solvent suggest that classical
molecular dynamics models coupled with ab initio calculations need parameterization of some of the core-core functions to provide accurate
description of intermolecular interactions due to the  difficulties to sample transitions from the molecular form to the contact ion pair 
form \citep{wick2017comparison}. These are indication that classical models parameterized with ab initio calculations  
are prone to problems of potential transferability and lead to poor description of intermolecular interactions and hydrogen bonding 
between solute and solvents if some of the electronic interactions are not optimized or parameterized. 
At the moment, the electronic structure calculations to give predictive quantitative description of the ground-state electronic properties of
solute-solvent systems such as solvated  hydrogen halides including excitation energies is a full quantum mechanics 
based on the combination of first principle molecular dynamics density functional theory (DFT) and many-body perturbation theory (MBPT) in the
$G_{0}W_{0}$ approximation. According to studies on the ionization potential and electron affinity of solute-solvent system, density functional 
calculations followed by the $G_{0}W_{0}$ approximation \citep{gaiduk2018electron,pham2017electronic,opalka2014ionization} yield accurate 
results in self-energy calculations, however, it lacks the ability to account for the presence of spin-dependence in coulombic interactions,
inaccurate in its application to large and complex systems, not self-consistent \citep{aryasetiawan1998gw} and computationally costly.
An alternative approach to the $G_{0}W_{0}$ approximation to probe the ground-state electronic properties of solute-solvent systems
to derive photoelectron spectra is the coupled-cluster method. Coupled-cluster formalism at the single and double (triple) (CCSD(T) levels 
\citep{bartlett2007coupled} on water clusters \citep{blase2016erratum} and implementations based on equation-of-motion at coupled-cluster single and double levels
(EOM-CCSD) \citep{musial2011multireference} on water clusters \citep{lange2018relation} have been applied to probe the ionization potential at the ground-state level.
These studies indicate that the use of DFT followed by CC approaches predicts binding energies of solvent systems accurately with small
errors compared to $G_{0}W_{0}$, which underestimate the ionization potential. In addition, prediction of the ionization potential
of water clusters based on approximate variant of EOM-CCSD \citep{dutta2017assessment,dutta2018lower} such as P-EOM-MBPT2 yield results that can be compared
to $G_{0}W_{0}$ \citep{lange2018relation}. Not only have the EOM-CCSD proved to be a signicant tool to derive the electronic properties of water clusters but
also shown to be a powerful approximation to accurately account for relativistic effects such as spin-orbit coupling in describing the electron affinity, ground- 
and excited-state properties of halogen species \citep{shee2018equation,bouchafra2018predictive}. 
Quite recently, \citep{peng2015energy} have developed a much more elaborate method to target core
excitation energies within the EOM-CCSD ansatz based on the core valence separation (CVS) approximation \citep{cederbaum1980many}. This approach is seen to be
provide a path that only includes excitations from the essential core orbitals when computing the core-level spectra. 
In this study, we present the use of a new approach in the DFT in the form the so-called frozen density embedding (FDE) method and  CVS-EOM-CCSD 
to provide accurate ground-state photoelectron spectra, including excitation energies of the interaction hydrogen chloride on ice surface. This complicated approach give reliable ground- and excited-state properties based on its partitioning of large systems into subsystems and
taking the effect of the environment on a subsystem of interest into account. Recent formulation of FDE have shown that FDE still possesses an exact framework that allows the DFT to be substituted with WFT for one subsystem (WFT-in-DFT) \citep{gomes2008calculation,hofener2013solvatochromic} or for all the
subsystem under consideration (WFT-in-WFT) \citep{hofener2012calculation} as well as the incorporation of coupled-cluster theory, making it possible to calculate the embedding potential for the ground state as well as the excited states at high accuracy and with less computational cost. 








\bibliography{con.bib}
\bibliographystyle{apalike}
\end{document}

